% Options for packages loaded elsewhere
\PassOptionsToPackage{unicode}{hyperref}
\PassOptionsToPackage{hyphens}{url}
%
\documentclass[
]{article}
\usepackage{amsmath,amssymb}
\usepackage{lmodern}
\usepackage{iftex}
\ifPDFTeX
  \usepackage[T1]{fontenc}
  \usepackage[utf8]{inputenc}
  \usepackage{textcomp} % provide euro and other symbols
\else % if luatex or xetex
  \usepackage{unicode-math}
  \defaultfontfeatures{Scale=MatchLowercase}
  \defaultfontfeatures[\rmfamily]{Ligatures=TeX,Scale=1}
\fi
% Use upquote if available, for straight quotes in verbatim environments
\IfFileExists{upquote.sty}{\usepackage{upquote}}{}
\IfFileExists{microtype.sty}{% use microtype if available
  \usepackage[]{microtype}
  \UseMicrotypeSet[protrusion]{basicmath} % disable protrusion for tt fonts
}{}
\makeatletter
\@ifundefined{KOMAClassName}{% if non-KOMA class
  \IfFileExists{parskip.sty}{%
    \usepackage{parskip}
  }{% else
    \setlength{\parindent}{0pt}
    \setlength{\parskip}{6pt plus 2pt minus 1pt}}
}{% if KOMA class
  \KOMAoptions{parskip=half}}
\makeatother
\usepackage{xcolor}
\usepackage[margin=1in]{geometry}
\usepackage{color}
\usepackage{fancyvrb}
\newcommand{\VerbBar}{|}
\newcommand{\VERB}{\Verb[commandchars=\\\{\}]}
\DefineVerbatimEnvironment{Highlighting}{Verbatim}{commandchars=\\\{\}}
% Add ',fontsize=\small' for more characters per line
\usepackage{framed}
\definecolor{shadecolor}{RGB}{248,248,248}
\newenvironment{Shaded}{\begin{snugshade}}{\end{snugshade}}
\newcommand{\AlertTok}[1]{\textcolor[rgb]{0.94,0.16,0.16}{#1}}
\newcommand{\AnnotationTok}[1]{\textcolor[rgb]{0.56,0.35,0.01}{\textbf{\textit{#1}}}}
\newcommand{\AttributeTok}[1]{\textcolor[rgb]{0.77,0.63,0.00}{#1}}
\newcommand{\BaseNTok}[1]{\textcolor[rgb]{0.00,0.00,0.81}{#1}}
\newcommand{\BuiltInTok}[1]{#1}
\newcommand{\CharTok}[1]{\textcolor[rgb]{0.31,0.60,0.02}{#1}}
\newcommand{\CommentTok}[1]{\textcolor[rgb]{0.56,0.35,0.01}{\textit{#1}}}
\newcommand{\CommentVarTok}[1]{\textcolor[rgb]{0.56,0.35,0.01}{\textbf{\textit{#1}}}}
\newcommand{\ConstantTok}[1]{\textcolor[rgb]{0.00,0.00,0.00}{#1}}
\newcommand{\ControlFlowTok}[1]{\textcolor[rgb]{0.13,0.29,0.53}{\textbf{#1}}}
\newcommand{\DataTypeTok}[1]{\textcolor[rgb]{0.13,0.29,0.53}{#1}}
\newcommand{\DecValTok}[1]{\textcolor[rgb]{0.00,0.00,0.81}{#1}}
\newcommand{\DocumentationTok}[1]{\textcolor[rgb]{0.56,0.35,0.01}{\textbf{\textit{#1}}}}
\newcommand{\ErrorTok}[1]{\textcolor[rgb]{0.64,0.00,0.00}{\textbf{#1}}}
\newcommand{\ExtensionTok}[1]{#1}
\newcommand{\FloatTok}[1]{\textcolor[rgb]{0.00,0.00,0.81}{#1}}
\newcommand{\FunctionTok}[1]{\textcolor[rgb]{0.00,0.00,0.00}{#1}}
\newcommand{\ImportTok}[1]{#1}
\newcommand{\InformationTok}[1]{\textcolor[rgb]{0.56,0.35,0.01}{\textbf{\textit{#1}}}}
\newcommand{\KeywordTok}[1]{\textcolor[rgb]{0.13,0.29,0.53}{\textbf{#1}}}
\newcommand{\NormalTok}[1]{#1}
\newcommand{\OperatorTok}[1]{\textcolor[rgb]{0.81,0.36,0.00}{\textbf{#1}}}
\newcommand{\OtherTok}[1]{\textcolor[rgb]{0.56,0.35,0.01}{#1}}
\newcommand{\PreprocessorTok}[1]{\textcolor[rgb]{0.56,0.35,0.01}{\textit{#1}}}
\newcommand{\RegionMarkerTok}[1]{#1}
\newcommand{\SpecialCharTok}[1]{\textcolor[rgb]{0.00,0.00,0.00}{#1}}
\newcommand{\SpecialStringTok}[1]{\textcolor[rgb]{0.31,0.60,0.02}{#1}}
\newcommand{\StringTok}[1]{\textcolor[rgb]{0.31,0.60,0.02}{#1}}
\newcommand{\VariableTok}[1]{\textcolor[rgb]{0.00,0.00,0.00}{#1}}
\newcommand{\VerbatimStringTok}[1]{\textcolor[rgb]{0.31,0.60,0.02}{#1}}
\newcommand{\WarningTok}[1]{\textcolor[rgb]{0.56,0.35,0.01}{\textbf{\textit{#1}}}}
\usepackage{graphicx}
\makeatletter
\def\maxwidth{\ifdim\Gin@nat@width>\linewidth\linewidth\else\Gin@nat@width\fi}
\def\maxheight{\ifdim\Gin@nat@height>\textheight\textheight\else\Gin@nat@height\fi}
\makeatother
% Scale images if necessary, so that they will not overflow the page
% margins by default, and it is still possible to overwrite the defaults
% using explicit options in \includegraphics[width, height, ...]{}
\setkeys{Gin}{width=\maxwidth,height=\maxheight,keepaspectratio}
% Set default figure placement to htbp
\makeatletter
\def\fps@figure{htbp}
\makeatother
\setlength{\emergencystretch}{3em} % prevent overfull lines
\providecommand{\tightlist}{%
  \setlength{\itemsep}{0pt}\setlength{\parskip}{0pt}}
\setcounter{secnumdepth}{-\maxdimen} % remove section numbering
\usepackage{booktabs}
\usepackage{longtable}
\usepackage{array}
\usepackage{multirow}
\usepackage{wrapfig}
\usepackage{float}
\usepackage{colortbl}
\usepackage{pdflscape}
\usepackage{tabu}
\usepackage{threeparttable}
\usepackage{threeparttablex}
\usepackage[normalem]{ulem}
\usepackage{makecell}
\usepackage{xcolor}
\usepackage{booktabs}
\usepackage{longtable}
\usepackage{array}
\usepackage{multirow}
\usepackage{wrapfig}
\usepackage{float}
\usepackage{colortbl}
\usepackage{pdflscape}
\usepackage{tabu}
\usepackage{threeparttable}
\usepackage{threeparttablex}
\usepackage[normalem]{ulem}
\usepackage{makecell}
\usepackage{xcolor}
\ifLuaTeX
  \usepackage{selnolig}  % disable illegal ligatures
\fi
\IfFileExists{bookmark.sty}{\usepackage{bookmark}}{\usepackage{hyperref}}
\IfFileExists{xurl.sty}{\usepackage{xurl}}{} % add URL line breaks if available
\urlstyle{same} % disable monospaced font for URLs
\hypersetup{
  pdftitle={Project 1},
  pdfauthor={Name: Partner:},
  hidelinks,
  pdfcreator={LaTeX via pandoc}}

\title{Project 1}
\author{Name:\\
Partner:}
\date{2023-04-09}

\begin{document}
\maketitle

{
\setcounter{tocdepth}{3}
\tableofcontents
}
\hypertarget{background}{%
\subsection{Background}\label{background}}

The World Health Organization has recently employed a new data science
initiative, \emph{CSIT-165}, that uses data science to characterize
pandemic diseases. \emph{CSIT-165} disseminates data driven analyses to
global decision makers.

\emph{CSIT-165} is a conglomerate comprised of two fabricated entities:
\emph{Global Health Union (GHU)} and \emph{Private Diagnostic
Laboratories (PDL)}. Your and your partner's role is to play a data
scientist from one of these two entities.

\hypertarget{data}{%
\subsection{Data}\label{data}}

\begin{quote}
\href{https://github.com/CSSEGISandData/COVID-19/tree/master/csse_covid_19_data/csse_covid_19_time_series}{2019
Novel Coronavirus COVID-19 (2019-nCoV) Data Repository by John Hopkins
CSSE} Data for 2019 Novel Coronavirus is operated by the John Hopkins
University Center for Systems Science and Engineering (JHU CSSE). Data
includes daily time series CSV summary tables, including confirmations,
recoveries, and deaths. Country/region are countries/regions hat conform
to World Health Organization (WHO). Lat and Long refer to coordinates
references for the user. Date fields are stored in MM/DD/YYYY format.
\end{quote}

\hypertarget{project-objectives}{%
\subsection{Project Objectives}\label{project-objectives}}

\hypertarget{objective-1}{%
\subsubsection{Objective 1}\label{objective-1}}

\begin{Shaded}
\begin{Highlighting}[]
\NormalTok{confirmed\_cases }\OtherTok{\textless{}{-}} \StringTok{"https://raw.githubusercontent.com/CSSEGISandData/COVID{-}19/master/csse\_covid\_19\_data/csse\_covid\_19\_time\_series/time\_series\_covid19\_confirmed\_global.csv"}
\NormalTok{covid\_deaths }\OtherTok{\textless{}{-}} \StringTok{"https://raw.githubusercontent.com/CSSEGISandData/COVID{-}19/master/csse\_covid\_19\_data/csse\_covid\_19\_time\_series/time\_series\_covid19\_deaths\_global.csv"}
\CommentTok{\#load data}
\NormalTok{cases\_df }\OtherTok{\textless{}{-}} \FunctionTok{read.csv}\NormalTok{(confirmed\_cases, }\AttributeTok{header =} \ConstantTok{TRUE}\NormalTok{, }\AttributeTok{na.strings =} \FunctionTok{c}\NormalTok{(}\StringTok{""}\NormalTok{, }\StringTok{" "}\NormalTok{))}
\NormalTok{deaths\_df }\OtherTok{\textless{}{-}} \FunctionTok{read.csv}\NormalTok{(covid\_deaths, }\AttributeTok{header =} \ConstantTok{TRUE}\NormalTok{, }\AttributeTok{na.strings =} \FunctionTok{c}\NormalTok{(}\StringTok{""}\NormalTok{, }\StringTok{" "}\NormalTok{))}

\CommentTok{\#segment first day of COVID data}
\NormalTok{data\_cases }\OtherTok{\textless{}{-}}\NormalTok{ dplyr}\SpecialCharTok{::}\FunctionTok{select}\NormalTok{(cases\_df, Province.State, Country.Region, X1.}\FloatTok{22.20}\NormalTok{); }
\NormalTok{data\_deaths }\OtherTok{\textless{}{-}}\NormalTok{ dplyr}\SpecialCharTok{::}\FunctionTok{select}\NormalTok{(deaths\_df, Province.State, Country.Region, X1.}\FloatTok{22.20}\NormalTok{)}

\CommentTok{\# Filter for the first day and select relevant columns}
\NormalTok{first\_day\_cases }\OtherTok{\textless{}{-}}\NormalTok{ cases\_df }\SpecialCharTok{\%\textgreater{}\%}
  \FunctionTok{filter}\NormalTok{(X1.}\FloatTok{22.20} \SpecialCharTok{!=} \DecValTok{0}\NormalTok{) }\SpecialCharTok{\%\textgreater{}\%}
  \FunctionTok{select}\NormalTok{(Province.State, Country.Region, X1.}\FloatTok{22.20}\NormalTok{)}

\NormalTok{first\_day\_deaths }\OtherTok{\textless{}{-}}\NormalTok{ deaths\_df }\SpecialCharTok{\%\textgreater{}\%}
  \FunctionTok{filter}\NormalTok{(X1.}\FloatTok{22.20} \SpecialCharTok{!=} \DecValTok{0}\NormalTok{) }\SpecialCharTok{\%\textgreater{}\%}
  \FunctionTok{select}\NormalTok{(Province.State, Country.Region, X1.}\FloatTok{22.20}\NormalTok{)}

\CommentTok{\# Identify the country with the highest confirmed cases and highest deaths}
\NormalTok{max\_cases }\OtherTok{\textless{}{-}}\NormalTok{ first\_day\_cases }\SpecialCharTok{\%\textgreater{}\%}
  \FunctionTok{filter}\NormalTok{(X1.}\FloatTok{22.20} \SpecialCharTok{==} \FunctionTok{max}\NormalTok{(X1.}\FloatTok{22.20}\NormalTok{)) }\SpecialCharTok{\%\textgreater{}\%}
  \FunctionTok{pull}\NormalTok{(Country.Region)}

\NormalTok{max\_deaths }\OtherTok{\textless{}{-}}\NormalTok{ first\_day\_deaths }\SpecialCharTok{\%\textgreater{}\%}
  \FunctionTok{filter}\NormalTok{(X1.}\FloatTok{22.20} \SpecialCharTok{==} \FunctionTok{max}\NormalTok{(X1.}\FloatTok{22.20}\NormalTok{)) }\SpecialCharTok{\%\textgreater{}\%}
  \FunctionTok{pull}\NormalTok{(Country.Region)}
\CommentTok{\# Determine if the country is the origin of the outbreak}
\ControlFlowTok{if}\NormalTok{(max\_cases }\SpecialCharTok{==}\NormalTok{ max\_deaths) \{}
\NormalTok{  output }\OtherTok{\textless{}{-}} \FunctionTok{paste}\NormalTok{(}\StringTok{"The origin of the COVID{-}19 outbreak was likely"}\NormalTok{, max\_cases)}
  \FunctionTok{print}\NormalTok{(output)}
\NormalTok{\}}
\end{Highlighting}
\end{Shaded}

\begin{verbatim}
## [1] "The origin of the COVID-19 outbreak was likely China"
\end{verbatim}

\hypertarget{objective-2-where-is-the-most-recent-area-to-have-a-first-confirmed-case}{%
\subsubsection{Objective 2: Where is the most recent area to have a
first confirmed
case?}\label{objective-2-where-is-the-most-recent-area-to-have-a-first-confirmed-case}}

\begin{Shaded}
\begin{Highlighting}[]
\CommentTok{\# iterates through each column that contains a date}
\ControlFlowTok{for}\NormalTok{(date\_column }\ControlFlowTok{in}\NormalTok{ (}\DecValTok{5}\SpecialCharTok{:}\FunctionTok{ncol}\NormalTok{(cases\_df)))\{}
  
  \CommentTok{\# iterates through each row (case count) for that specific date}
  \ControlFlowTok{for}\NormalTok{(x }\ControlFlowTok{in}\NormalTok{ (}\DecValTok{1}\SpecialCharTok{:}\FunctionTok{length}\NormalTok{(cases\_df[,date\_column])))\{ }\CommentTok{\# subsets the column for a single date}
    \ControlFlowTok{if}\NormalTok{(cases\_df[x, date\_column] }\SpecialCharTok{==} \DecValTok{1} \SpecialCharTok{\&}\NormalTok{ cases\_df[x, date\_column}\DecValTok{{-}1}\NormalTok{] }\SpecialCharTok{==} \DecValTok{0}\NormalTok{)\{ }\CommentTok{\# checks if there is a new case }
\NormalTok{      newest\_case }\OtherTok{\textless{}{-}}\NormalTok{ cases\_df[x, }\DecValTok{2}\NormalTok{] }\CommentTok{\# updates variable with the corresponding country name (column 2) }
\NormalTok{    \}}
\NormalTok{  \}}
\NormalTok{\}}

\FunctionTok{cat}\NormalTok{(}\StringTok{"The most recent area to have a first confirmed case is"}\NormalTok{, newest\_case)}
\end{Highlighting}
\end{Shaded}

\begin{verbatim}
## The most recent area to have a first confirmed case is Korea, North
\end{verbatim}

\hypertarget{objective-3}{%
\subsubsection{Objective 3}\label{objective-3}}

\hypertarget{objective-4}{%
\subsubsection{Objective 4}\label{objective-4}}

\hypertarget{objective-4.1}{%
\paragraph{Objective 4.1}\label{objective-4.1}}

\hypertarget{objective-4.2}{%
\paragraph{Objective 4.2}\label{objective-4.2}}

\hypertarget{github-log}{%
\subsubsection{GitHub Log}\label{github-log}}

\begin{Shaded}
\begin{Highlighting}[]
\FunctionTok{git}\NormalTok{ log }\AttributeTok{{-}{-}pretty}\OperatorTok{=}\NormalTok{format:}\StringTok{"\%nSubject: \%s\%nAuthor: \%aN\%nDate: \%aD\%nBody: \%b"}
\end{Highlighting}
\end{Shaded}

\begin{verbatim}
## 
## Subject: Completed Objective 2 using confirmed cases df, added comments
## Author: PreenaM
## Date: Sun, 9 Apr 2023 17:00:51 -0700
## Body: 
## 
## Subject: adding *Morgan's* progress on Objective 1 from previous repo
## Author: PreenaM
## Date: Sun, 9 Apr 2023 11:44:24 -0700
## Body: 
## 
## Subject: wget CSV for deaths (GHU)
## Author: PreenaM
## Date: Sun, 9 Apr 2023 11:40:55 -0700
## Body: 
## 
## Subject: wget CSV file for confirmed cases (PDL)
## Author: PreenaM
## Date: Sun, 9 Apr 2023 11:40:37 -0700
## Body: 
## 
## Subject: Added template
## Author: PreenaM
## Date: Sun, 9 Apr 2023 10:55:49 -0700
## Body: 
## 
## Subject: Updated README with team member names
## Author: PreenaM
## Date: Sun, 9 Apr 2023 10:40:48 -0700
## Body: 
## 
## Subject: Initial commit
## Author: PreenaM
## Date: Sun, 9 Apr 2023 10:34:23 -0700
## Body:
\end{verbatim}

\end{document}
